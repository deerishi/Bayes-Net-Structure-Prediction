\section{Introduction}

The purpose of this project is to explore the area of Structure Prediction. Knowing the underlying structure  of a dataset can significantly help in the area of supervised learning and feature Engineering. This project explores the gains in performance of supervised learning when the underlying Bayesian Network structure of a dataset is known and is used to do a Bayes Net inference for a classification task. Its performance has been compared to that of Document based Naive Bayes Model.

\subsection{What is the application domain?}

The application domain is primarily structure prediction for supervised learning problems (both categorical and continuous data). The algorithm used can be applied to a wide variety of datasets -$text,music,images$ and $time$ $series$ $data$.

\subsection{What is the problem?}

General Purpose Machine Learning Tasks like Image Classification, Sentiment Analysis can be solved with a very high accuracy using Deep Learning. However, the prerequisite of deep learning models is the availability of a large amount of labelled data. While, images and general purpose text (like newsgroups, e.t.c.) are publically available, when making a machine learning model for a custom specific task (say document categorization according to a very specific criteria ), we may not have access to a large amount of labelled data.
Using Deep learning models is very difficult in such situations. Manual feature engineering is required when we have limited access to labelled data. Although, feature  engineering requires a certain amount of domain knowledge, even with that one has to experiment with a number of possible different combinations of features that tend to work out the best.

In such a case, knowing the structure of a dataset that describes how different features are dependent on one another can prove to be really helpful.

In cases where the structure of a dataset is not known, Naive Bayes Algorithm is applied. Although in many cases it gives good results (Email Spam classification, Text categorization, etcetera) , it does not take into account the different probabilistic dependencies between the features.
This project aims to first infer those probabilistic dependencies between the features and exploit them in classification problems.

\vspace{60.6pt}




